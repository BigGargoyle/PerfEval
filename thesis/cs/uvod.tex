\chapter*{Úvod}
\addcontentsline{toc}{chapter}{Úvod}
Po~více než dvacet let pomáhají unit testy udržovat kvalitu kódu v~průběhu vývoje
softwaru. Za~tuto dobu bylo vyvinuto spoustu knihoven pro~implementaci a~spouštění
unit testů. Verzovací nástroje jako GitLab, nebo GitHub umožňují v~rámci vývoje
software verzovat, spouštět unit testy a~reagovat na~jejich případná selhání.

Pro~udržování korektnosti kódu využíváme pokrytí unit testy. Udržování korektnosti
je dnes již naprosto běžnou součástí vývoje softwaru. Psaní výkonnostních testů ovšem
tak běžné není. Práce s~frameworky pro~měření výkonu totiž není tak jednoduchá jako používání
knihoven pro~psaní unit testů, a~proto jejich užití není tak časté.

Frameworky pro~vyhodnocování výkonu softwaru jsou podobné knihovnám pro psaní unit testů.
Frameworky jako JMH nebo BenchmarkDotNet pomáhají implementovat výkonnostní testy.
Výkonnostní testy se implementují obdobně jako u~unit testů obvykle pomocí anotací.
Vyhodnocování naměřených výsledků dnes obvykle zahrnuje i~ruční zhodnocení naměřených výsledků.
Vyhodnocování výkonu totiž vyžaduje porovnat naměřený výkon s~nějakou další referenční hodnotou.
Měřící frameworky tyto referenční hodnoty nemají a~ani je nikde získat nemohou,
protože měří pouze jednu verzi softwaru. Nemohou tedy dělat zmíněné celkové vyhodnocení.

Z~porovnání vyhodnocování unit testů a~výkonnostních testů, není vyhodnocování
výkonnostních testů tak jednoduché. Unit testy testují korektnost. Korektnost
se prokáže tak, že~na~každý zadaný vstup program vrátí očekávaný výstup. Při vyhodnocování
výkonu se~změří nějaká data o~běhu programu. Tato data ale sama o~sobě nic neříkají
a~je nutné je zkoumat v kontextu.

Výsledkem této práce je nástroj nazvaný PerfEval. PerfEval je konzolová aplikace
napsaná v~programovacím jazyce Java. Perfeval umí pomocí argumentů na~příkazové
řádce vyhodnocovat výsledky výkonnostních testů.

PerfEval je schopen automatického hlášení výsledků výkonnostních testů. Umí porovnávat
výsledky měření výkonu dvou verzí softwaru mezi sebou. PerfEval je také vhodný pro
skriptování, protože o~výsledcích informuje nejen výpisem na~standardní výstup, ale
také exit kódem. Nástroj podporuje zpracování výstupů frameworků BenchmarkDotNet a~JMH ve~formátu JSON,
ale je rozšiřitelný i~pro~zpracování jiných frameworků, nebo formátů.
