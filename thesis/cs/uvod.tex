\chapter{Úvod}
%\addcontentsline{toc}{chapter}{Úvod}

Po~více než dvacet let pomáhají unit testy udržovat kvalitu kódu v~průběhu vývoje
softwaru. Za~tuto dobu bylo vyvinuto mnoho knihoven pro~implementaci a~spouštění
unit testů. Verzovací nástroje jako GitLab nebo GitHub umožňují v~rámci vývoje
software verzovat, spouštět unit testy a~reagovat na~jejich případná selhání.

Psaní výkonnostních testů, které by pomáhaly udržovat kvalitu kódu stejně jako unit testy,
již tak běžné není. Práce s~frameworky pro měření výkonu totiž není tak jednoduchá,
jako používání knihoven pro psaní unit testů. Průběžné udržování výkonnosti by však mohlo
pomáhat udržovat kvalitu kódu obdobným způsobem jako unit testy.

Některé frameworky pro~vyhodnocování výkonu softwaru jsou podobné frameworkům pro psaní unit testů.
Frameworky jako JMH nebo BenchmarkDotNet pomáhají implementovat výkonnostní testy.
Vyhodnocování naměřených výsledků dnes obvykle zahrnuje i~jejich ruční zhodnocení.
Vyhodnocování výkonu totiž vyžaduje porovnat naměřený výkon s~nějakou další referenční hodnotou.
Měřící frameworky tyto referenční hodnoty nemají a~ani je nikde získat nemohou,
protože měří pouze jednu verzi softwaru. Nemohou tedy zmíněné celkové vyhodnocení dělat.

Vyhodnocování výkonnostních testů je složitější problém než vyhodnocování unit testů.
Unit testy testují korektnost. Ta se prokáže tak, že~na~každý zadaný vstup program vrátí očekávaný výstup. Při vyhodnocování
výkonu se~změří sada dat o~běhu programu. Tato data ale sama o~sobě nemají požadovanou
vypovídající hodnotu a je nutné je zkoumat v kontextu.

Výsledkem této práce je nástroj nazvaný PerfEval. PerfEval je konzolová aplikace
napsaná v~programovacím jazyce Java. PerfEval umí vyhodnocovat výsledky
výkonnostních testů. Způsob a~průběh vyhodnocování je řízený pomocí argumentů příkazové řádky.

PerfEval je schopen automatického hlášení výsledků výkonnostních testů. Umí porovnávat
výsledky měření výkonu dvou verzí softwaru mezi sebou. PerfEval je také vhodný pro
skriptování, protože o~výsledcích informuje nejen výpisem na~standardní výstup, ale
také exit kódem. Nástroj podporuje zpracování výstupů frameworků BenchmarkDotNet a~JMH ve~formátu JSON,
ale je rozšiřitelný i~pro~zpracování jiných frameworků nebo formátů.
