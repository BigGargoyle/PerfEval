\chapter{Programátorská dokumentace systému PerfEval}

\section{Architektura systému}

\section{Rozšiřitelnost a její omezení}

Systém PerfEval byl od~počátku projektován tak, aby byl rozšiřitelný. Hlavní jádro
celé aplikace tvoří vyhodnocování výkonu. V~různých částech návrhu se~vyskytují místa,
kde je možné významným způsobem doplnit a~změnit chování celé aplikace.

Nejdůležitější ze~zmiňovaných rozšíření je rozšíření o~datový formát. Tato možnost dělá z~PerfEvalu
poměrně univerzální nástroj. Činí ho totiž méně závislým na~použitém měřícím systému a~výstupním jeho formátu.

Rozšiřitelnost systému PerfEval není neomezená. Rozšíření, která nebudou zmíněná v~této kapitole,
pravděpodobně nebudou možná, nebo budou vyžadovat mnohem více času pro~vývoj. Naopak pro~rozšíření
v~této kapitole existuje návod jak systém rozšířit.

\subsection{Rozšíření o datový formát}

Vezměmě opět příklad našeho programátora z~první kapitoly. Programátor si napsal výkonnostní testy.
Testy napsal pomocí nástroje, který nepodporuje PerfEval. Nicméně progrmátor ví, že~systém PerfEval
dělá přesně to, co~potřebuje. Jediný problém je tedy ve~vysvětlení svého datového formátu systému.

PerfEvalje od~počátku zamýšlen pro rozšíření v~tomto místě. Programátor tedy musí prozkoumat, jak správně zkonstruovat
třídu Samples. Třída Samples totiž reprezentuje jedno spuštění testovací metody měřícího frameworku. Všechny hodnoty
naměřené jednou metodou jsou tedy uloženy zde.

Programátor musí naimplementovat vlastní implementaci rozhraní MeasurementParser. V~tomto rozhraní je důležitá metoda
getTestsFromFiles. Tato metoda na~vstupu přijme všechny soubory s~výsledky měření jedné verze. Výstupem je list objektů
typů Samples. Pro každou metodu (měřící test), který se v~souborech nachází, se v~listu vyskytuje pouze jedna instance
typu Samples.

Pokud jsem tedy měření pomocí frameworku spustil dvakrát se~stejnými měřícími metodami, tak se ve~výsledných Samples
každá objeví právě jednou. Naměřené hodnoty budou všechny u~této jedné instance Samples.

Poslední krok tohoto rozšíření po~naimplementování rozhraní MeasurementParser je jeho registrace. Registrace probíhá tak,
že~se~přidá jeden řádek do~statického konstruktoru třídy ParserFactory. Na~řádku bude přidání položky do~objektu
s~názvem registeredParsers. Tento objekt je typu HashMap a~přiřazuje k~sobě Supplier, který vrací MeasurementParser
a~název typu String. Přidávaná položka je tedy String odpovídající názvu parseru a~reference na~metodu,
která umí parser zkonstruovat.

Použití nového parseru je pak jednoduché. Při inicializaci systému PerfEval příkazem init se jako parametr argumentu
benchmark-parser použije jméno nového parseru. Dokonce je začne hlásit v~nabídce i~varování v~případě, že žádný parser
není při~inicializaci zadán.

\subsection{Rozšíření o filtr}

Pokud uživatel PerfEvalu chce změnit pořadí výpisu testů~na výstupu, může implementovat nový filtr.
Tento filtr objektem typu Comparator, který porovnává instance MeasurementComparisonRecord. Pomocí tohoto
komparátoru, pak dojde k~setřídění vypisovaných prvků.

Dále je nutné přepsat metodu resolvePrinterForEvaluateCommand takovým způsobem, aby~používala i~nový druh filtru.
Tuto metodu je možné nalézt ve~třídě SetupUtilities. Komparátor se~pak může předávat objektům typu ResultPrinter
při konstrukci. O~jejich dalším použití si tedy tyto objekty rozhodují samy.

\subsection{Rozšíření o statistický test}

Může se stát, že~uživatel systému PerfEval má o~svých datech nějaké informace, které
by chtěl při~vyhodnocování zohlednit. Může si tedy naprogramovat vlastní implementaci
rozhraní StatisticTest.

Po~naprogramování vlastní implementace rozhraní StatisticTest, pak jen stačí ve~třídě SetupUtilities
změnit chování metody resolveStatisticTest, tak~aby~brala novou implementaci v~potaz. Posledním krokem je
přidání nového flagu do~parseru příkazové řádky v~metodě createParser.

\subsection{Rozšíření o možnost výpisu}

Pro vypisování výsledků porovnání slouží rozhraní ResultPrinter. Pokud by si uživatel chtěl naimplementovat
vlastní způsob vypisování, tak stačí impleemtnovat jedoinou jeho metodu PrintResults.

Přidání nového ResultPrinteru je podobné jeko u~přidávání nové implementace rozhraní StatisticTest.
Stačí změnit implementaci metody resolvePrinterForEvaluateCommand ve~třídě SetupUtilities.
Úprava by~měla být provedena tak, aby~metoda uměla vrátit i~novou implementaci.
Pokud by~bylo zapotřebí nového argumentu na~příkazové řádce, takje nutné jej také
správně naimplementovat do~metody createParser.

\subsection{Změna použitého databázového systému}

V~důsledku dlouhého rozhodování se o~tom, jak se budou informace o~výsledcích měření ukládat vzniklo rozhraní Database.
Rozhraní má spoustu metod. Pokud by se uživatel rozhodl změnit způsob ukládání dat o~měřeních, tak by
musel implementovat celé toto rozhraní. Po~implementaci rozhraní pak už jen stačí změnit metodu constructDatabase ve~třídě
SetupUtilities, která vrací instanci objektu typu Database.

\subsection{Rozšíření o příkaz}

Každý příkaz PerfEval se skládá ze~dvou tříd. Jedná se o~třídu implementující rozhraní CommandSetup
a~o~třídu implementující rozhraní Command. Pokud by uživatel chtěl doplnit nějaký nový příkaz,
který by v~kontextu systému dával smysl, tak je to možné. Dobrý příklad pro~reprezentování nového příkazu
bude vyhodnocení výkonu s~grafickým výstupem.

Na~rozdíl od~stávajícího vyhodnocování by~grafické vyhodnocování potřebovalo údaje z~více než dvou posledních verzí.
Proto by příkaz evaluate-graphical třída rozpoznala jako a~vytvořila instanci třídy EvaluateGraphicalSetup.
Na~této třídě, implementaci CommandSetup, by pak zavolala metodu setup. Metoda setup na~třídě EvaluateGraphicalSetup
by měla za~úkol z~konfigurace PerfEvalu a~příkazové řádky vyrobit objekt EvaluateGraphicalCommand. Tento objekt, který
by implementoval rozhraní Command by pak metoda getCommand na~třídě Parser vrátila.

Zbytek programu by se již nezměnil metoda main ve~třídě Main by vykonala dodaný příkaz. Spustila by~standardním způsobem
metodu execute na~instanci objektu typu Command.

Zaregistrování nového příkazu by probíhalo přidáním nové položky do~statického konstruktoru podobně.
Položka mapy commandPerSetup má obdobnou strukturu jako v~případě rozšíření o~nový MeasurementParser.
Dodal by se řádek s~názvem příkazu typu String a~z~reference na~bezparametrický konstruktor
implementace rozhraní CommandSetup. Název příkazu odpovídá příkazu, kterým se bude volat z~příkazové řádky.

\subsection{Omezená rošiřitelnost ve vyhodnocování}

Jeden z nejhorších požadavků na rozšíření systému by bylo rozšíření v oblasti vyhodnocování.
Jedná se zejména o změnu implementace třídy PerformanceEvaluator. Tato třída utváří
celkovou vyhodnocovací logiku. Její rozšiřitelnost je omezená na impelmentace rozhraní
StatisticTest.

Změnu chování vyhodnocování výkonnostních testů lze ale provést. Omezené změny ve vyhodnocování
se provádí změnami hodnot uvnitř config.ini souboru.