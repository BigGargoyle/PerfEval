\chapter{Programátorská dokumentace systému PerfEval}

\section{Architektura systému}

Architekturu systému PerfEval je možné rozdělit do~dvou částí. První část systému tvoří
parser příkazové řádky a~tzv. setup třídy. Druhou částí jsou tzv. command třídy,
které provádí skutečně požadovanou činnost. V~této části dokumentace nemusí být
některé podrobnosti o~implementaci, které je možné najít v~automaticky generované dokumentaci
JavaDoc. Jedná se například o~argumenty zmiňovaných metod.

Vnitřní struktura běhu je velice jednoduchá. Metoda main má tři jednoduché úkoly.
Zavolat metodu getCommand na~třídě Parser. Tato metoda vrátí objekt typu Command,
na kterém metoda main zavolá metodu execute. Metoda execute při návratu vrací enumerátor
typu ExitCode. Na~objektu ExitCode pak na~konci metoda main zavolá metodu exit,
která ukončuje program s~požadovaným exit kódem. Metody getCommand a~execute mohou skončit
s~výjimkami ParserException a~PerfEvalCommandFailedException. Pro tyto případy tyto
výjimky obsahují položku exitCode na~které metoda main opět zavolá metodu exit.

\subsection{Parser a setup třídy}

Hlavní část třídy Parser tvoří slovník commandPerSetup a~metoda getCommand.
Ve slovníku jsou k jednotlivým řetězcovým klíčům přiřazeny objekty typu
Supplier, které mají za~úkol vracet instance typu CommandSetup.
Klíče jsou skutečné řetězce zadávané uživatelem. Parser pak na~základě příkazu
zadaného na~příkazové řádce zkonstruuje skrze Supplier získaný ze~slovníku správnou
instanci třídy CommandSetup. Na~této instanci se zavolá metoda setup, která vrací
objekt typu Command, podle definice rozhraní CommandSetup. Metoda getCommand je volaná
z~metody main, která tvoří hlavní metodu programu.

Třída Parser předává do~metody setup objekty OptionSet a~PerfEvalConfig. Třída PerfEvalConfig
reprezentuje globální konfiguraci systému PerfEval podle konfiguračního souboru.
Parser tedy skrze statické metody této třídy zařídí přečtení konfiguračního souboru
do~objektu PerfEvalConfig. Objekt OptionSet reprezentuje parametry a~značky zadané
uživatelem na~příkazovou řádku. Jedná se o~objekt z~knihovny joptsimple. Rozpoznávány jsou
všechny značky blíže specifikované ve~třídě SetupUtilities.

SetupUtilities je třída obsahující statické položky a~řetězcové konstanty.
Tyto položky a~konstanty slouží pro~zpracování argumentů příkazové řádky.
Setup třídy a~Parser využívají jednotlivé metody této třídy.
Protože některé setup třídy využívají stejných metod ze~třídy SetupUtilities, tak jsou umístěny
právě ve~třídě SetupUtilities.

InitSetup je třída, která má za~úkol připravit instanci InitCommand. Protože se jedná
o~setup třídu, tak shání informace pro~práci Commandu, který konstruuje. Poté, co v~rámci
metody setup sežene všechny potřebné informace zkonstruuje InitCommand a~vrátí jej.
Tímto je InitCommand připraven k~práci.

IndexNewSetup je třída pro přípravu instance AddFileCommand. IndexNewSetup dodává
při konstrukci AddFileCommandu implementaci rozhraní Database, cestu k přidávanému souboru
a~instanci objektu ProjectVersion. ProjectVersion popisuje verzi softwaru pro kterou
byl nově přidávaný výsledek změřen. Zkonstruovanou instanci třídy AddFileCommand vrací metoda setup.

IndexAllSetup je třída pro přípravu instance AddFilesFromDirCommand. IndexAllSetup dodává
při konstrukci AddFilesFromDirCommandu implementaci rozhraní Database, cestu ke složce, ze které se budou přidávat soubory,
a~instanci objektu ProjectVersion. ProjectVersion popisuje verzi softwaru pro kterou
byly nově přidávané výsledky změřeny. Zkonstruovanou instanci třídy AddFilesFromDirCommand vrací metoda setup.

EvaluateSetup je třída která připravuje EvaluateCLICommand. Musí v~databázi najít
soubory s~výsledky měření, které bude EvaluateCLICommand porovnávat. Dále vyrábí
implementace rozhraní StatisticTest a ResultPrinter, podle příznaků z~příkazové řádky.
EvaluateSetup musí také vyrobit instanci objektu PerformanceEvaluator. Tuto instanci vyrobí
z~nastavení v~objektu PerfEvalConfig a~ze~zmíněné implementace StatisticTest. EvaluateCLICommand je zkonstruovaný
z~vyrobeného PerformanceEvaluatoru, ResultPrinteru, vstupních souborů s~výsledky
a~MeasurementParseru. MeasurementParser je součástí objektu PerfEvalConfig.

UndecidedSetup konstruuje EvaluateCLICommand způsobem z~předchozího odstavce.
Jediný rozdíl je, že se použije pevně stanovený ResultPrinter typu UndecidedPrinter.
Tato implementace CommandSetup vznikla proto, aby se vypsání informace o~nerozhodných
testech vypisovala pomocí odlišného příkazu.

Třída ListResultsSetup připravuje ListResultsCommand. Jediná činnost metody setup
spočívá ve~vyrobení instance rozhraní Database. Tuto instanci předá konstruktoru
třídy ListResultsCommand. Vyrobenou instanci třídy ListResultsCommand vrací.

\subsection{Command třídy}

PerfEval je systém ovládaný pomocí příkazů (commandů) z~příkazové řádky. Každému z~příkazů
odpovídá některá z~implementací rozhraní Command. Implementace třídy Command musejí mít implementovanou
jedinou metodu execute.

Třída AddFileCommand má~za~úkol přidat nový výsledek měření do~databáze. Vnitřní položky
třídy jsou fileToAdd, database a~version. Metoda execute tedy volá metodu addFile na dodané implementaci
databáze. Předávané paramtery jsou fileToAdd a~version.

Třída AddFilesFromDirCommand má~za~úkol přidat všechny výsledky měření ve složce do~databáze. Vnitřní položky
třídy jsou sourceDirPath, database a~version. Metoda execute tedy volá metodu addFilesFromDir na~dodané implementaci
databáze. Předávané paramtery jsou sourceDirPath a~version.

%% nutno upravit implementaci -> detail
Třída InitCommand má za~úkol v~adresáři, odkud byl PerfEval z~příkazové řádky spuštěn, inicializovat systém PerfEval.
Inicializace systému PerfEval probíhá v několika krocích. Nejprve se vytvoří adresář .performance. Následně se do~.gitignore
(v případě neexistence bude vytvořen) souboru doplní ignorace souborů a adresářů systému PerfEval. Nakonec je~z~parametrů
dodaných při~konstrukci a~z~výchozích parametrů vytvořen konfigurační soubor config.ini. V případě, že systém je
v~adresáři již inicializovaný a nebyl dodán příznak násilné inicializace, tak systém nebude inicializovaný.

Třída EvaluateCLICommand má za~úkol porovnat výsledky měření výkonu dvou různých verzí.
Soubory s~výsledky dvou verzí a~parser k~nim jsou dodané při~konstrukci. Při~konstrukci
je dodaný objekt implementace rozhraní StatisticTest. Zmíněný objekt slouží ke~statistickému
porovnání výsledků měření. Při~konstrukci je dále dodaná implementace rozhraní ResultPrinter.
Implementace tohoto rozhraní rozhoduje o tom, jakým způsobem budou prezentovány výsledky porovnání.
Třída v rámci volání metody execute nechá naparsovat soubory s výsledky měření do objektů typu Samples.
Seznamy těchto objektů nechá porovnat pomocí statistického testu. Výsledkem je objekt MeasurementComparisonResultCollection.
Tento objekt je nakonec pomocí dodané implementace rozhraní ResultPrinter prezentováno uživateli.

Třída ListResultsCommand a~její metoda execute slouží k~prostému vypsání obsahu databáze s~výsledky měření.
Databáze obsahuje informace o~výsledcích měření jako je cesta k~souboru a~popis verze. Tyto údaje vypíše
metoda print na~třídě FileInfoPrinter do~přehledné tabulky na~standardní výstup.

\subsection{Implementace rozhraní MeasurementParser}

Rozhraní MeasurementParser je určeno ke~zpracování souborů s~výsledky měření výkonu.
Třída MeasurementFactory vrací správnou instanci MeasurementParseru na~základě dodaného jména (řetězce).
Jméno parseru je uloženo v konfiguračním souboru.

Implementace rozhraní mohou při zpracovávání souborů vyhazovat runtime výjimku
MeasurementParserException. Výjimka je odchytávaná mimo implementaci MeasurementParser
v~metodě execute třídy EvaluateCLICommand. Výjimku je tedy bezpečné vyhazovat.

Aktuálně dostupné implementace rozhraní MeasurementParser jsou JMHJSONParser a~BenchmarkDotNetJSONParser.
JMHJSONParser umí zprácovavat výstup frameworku JMH ve~formátu JSON. BenchmarkDotNetJSONParser
umí zprácovavat výstup frameworku BenchmarkDotNet ve~formátu JSON.

\subsection{Implementace rozhraní StatisticTest}

Rozhraní StatisticTest definuje jaké metody mají mít implementace statistických testů
pro porovnání dvourozmětných polí typu double. Jedná se o~metodu, která vrací interval spolehlivosti.
Interval spolehlivosti je intervalový odhad střední hodnoty pro rozdíl dvou náhodných veličin.
Dále musí umět vrátit odhad minimálního počtu vzorků, které jsou zapotřebí pro zajištění dostatečně úzkého intervalu.

Implementace tohoto rozhraní jsou dvě. První je třída ParametricTest, která používá
Welchův T-test, který je podrobněji popsán v~kapitole 2.3.1. Druhou implementací je
NonparamtericTest, který využívá hierarchického bootstrapu. Tento bootstrap je podrobněji
popsaný v~kapitole 2.3.2.

\subsection{Implementace rozhraní Database}

Rozhraní Database definuje funkce, které jsou požadovány od~systému, který má uládat
metadata o~výsledcích měření. Jeho jediná implementace využívá technologie H2 embedded databáze.
Technologie umožňuje vyhledávat položky v~lokální databázi pomocí standardních SQL příkazů.

\subsection{Struktura konfiguračního souboru}

Konfigurační soubor systému PerfEval se nachází ve~složce .performance a~má název config.ini.
V~souboru jsou dvě sekce. Jedná se o~sekce statistic a~project.

V~sekci statistic je možné nastavit
pět různých paramterů. Prvním parametrem je critValue, který odpovídá velikosti chyby I.~druhu při
statistickém testu. K~přímému ovlivnění statistických testů slouží ještě parametr maxCIWidth a~tolerance.
Parametr maxCIWidth říká, jaká může být relativní šířka intervalu spolehlivosti při nevyvrácení hypotézy, aby test prošel
s~kladným výsledkem. Parametr tolerance říká, o~kolik může klesnout výkon v~poměru s~předchozí verzí.
Parametr maxTestCount udává kolik je možné provést maximálně měření jedné verze. Parametr minTestCount
udává kolik minimálně by~mělo být dodaných naměřených výsledků (celkový počet běhů)
pro porovnání výkonů.

V~sekci project je možné nastavit používání git repozitáře (parametr git) pomocí hodnot TRUE a~FALSE.
Dalším parametrem je parserName. Tento parametr definuje parser, kterou bude vracet ParserFactory.

\section{Rozšiřitelnost a její omezení}

Systém PerfEval byl od~počátku projektován tak, aby byl rozšiřitelný. Hlavní jádro
celé aplikace tvoří vyhodnocování výkonu. V~různých částech návrhu se~vyskytují místa,
kde je možné významným způsobem doplnit a~změnit chování celé aplikace.

Nejdůležitější ze~zmiňovaných rozšíření je rozšíření o~datový formát. Tato možnost dělá z~PerfEvalu
poměrně univerzální nástroj. Činí ho totiž méně závislým na~použitém měřícím systému a~jeho výstupním formátu.

Rozšiřitelnost systému PerfEval není neomezená. Rozšíření, která nebudou zmíněná v~této kapitole,
pravděpodobně nebudou možná, nebo budou vyžadovat mnohem více času pro~vývoj. Naopak pro~rozšíření
v~této kapitole existuje návod jak systém jednoduše rozšířit.

\subsection{Rozšíření o datový formát}

Vezměmě opět příklad našeho programátora z~první kapitoly. Programátor si napsal výkonnostní testy.
Testy napsal pomocí nástroje, který nepodporuje PerfEval. Nicméně progrmátor ví, že~systém PerfEval
dělá přesně to, co~potřebuje. Jediný problém je tedy ve~vysvětlení svého datového formátu systému.

PerfEval je od~počátku zamýšlen pro rozšíření v~tomto místě. Programátor tedy musí prozkoumat, jak správně zkonstruovat
třídu Samples. Třída Samples totiž reprezentuje jedno spuštění testovací metody měřícího frameworku. Všechny hodnoty
naměřené jednou metodou jsou tedy uloženy zde.

Programátor musí naimplementovat vlastní implementaci rozhraní MeasurementParser. V~tomto rozhraní je důležitá metoda
getTestsFromFiles. Tato metoda na~vstupu přijme všechny soubory s~výsledky měření jedné verze. Výstupem je list objektů
typů Samples. Pro každou metodu (měřící test), který se v~souborech nachází, se v~listu vyskytuje pouze jedna instance
typu Samples.

Pokud jsem tedy měření pomocí frameworku spustil dvakrát se~stejnými měřícími metodami, tak se ve~výsledných Samples
každá objeví právě jednou. Naměřené hodnoty budou všechny u~této jedné instance Samples.

Poslední krok tohoto rozšíření po~naimplementování rozhraní MeasurementParser je jeho registrace. Registrace probíhá tak,
že~se~přidá jeden řádek do~statického konstruktoru třídy ParserFactory. Na~řádku bude přidání položky do~objektu
s~názvem registeredParsers. Tento objekt je typu HashMap a~přiřazuje k~sobě Supplier, který vrací MeasurementParser
a~název typu String. Přidávaná položka je tedy String odpovídající názvu parseru a~reference na~metodu,
která umí parser zkonstruovat.

Použití nového parseru je pak jednoduché. Při inicializaci systému PerfEval příkazem init se jako argumentu
benchmark-parser použije jméno nového parseru. Dokonce jej začne hlásit v~nabídce i~chybová hláška v~případě, že žádný parser
není při~inicializaci zadán.

\subsection{Rozšíření o filtr}

Pokud uživatel chce  PerfEvalu změnit pořadí výpisu testů~na výstupu, může použít filtr.
Pokud mu žádný z~připravených nevyhovuje, tak může implementovat nový filtr.
Tento filtr je objektem typu Comparator, který porovnává instance MeasurementComparisonRecord. Pomocí tohoto
komparátoru, pak dojde k~setřídění vypisovaných prvků.

Dále je nutné přepsat metodu resolvePrinterForEvaluateCommand takovým způsobem, aby~rozponávala i~nový druh filtru
podle příkazové řádky. Tuto metodu je možné nalézt ve~třídě SetupUtilities.
Posledním krokem je přidání nové značky do~parseru příkazové řádky v~metodě createParser.
Komparátor se~pak může předávat objektům typu ResultPrinter při~konstrukci.
O~jejich dalším použití si tedy tyto objekty rozhodují samy.

\subsection{Rozšíření o statistický test}

Může se stát, že~uživatel systému PerfEval má o~svých datech nějaké informace, které
by chtěl při~vyhodnocování zohlednit. Může si tedy naprogramovat vlastní implementaci
rozhraní StatisticTest.

Po~naprogramování vlastní implementace rozhraní StatisticTest, pak jen stačí ve~třídě SetupUtilities
změnit chování metody resolveStatisticTest, tak~aby~rozpoznnávala novou implementaci podle příkazové řádky. Posledním krokem je
přidání nové značky do~parseru příkazové řádky v~metodě createParser.

\subsection{Rozšíření o možnost výpisu}

Pro vypisování výsledků porovnání slouží rozhraní ResultPrinter. Pokud by si uživatel chtěl naimplementovat
vlastní způsob vypisování, tak stačí implmetnovat jedinou jeho metodu PrintResults.

Přidání nového ResultPrinteru je podobné jako u~přidávání nové implementace rozhraní StatisticTest.
Stačí změnit implementaci metody resolvePrinterForEvaluateCommand ve~třídě SetupUtilities.
Úprava by~měla být provedena tak, aby~metoda uměla vrátit i~novou implementaci.
Pokud by~bylo zapotřebí nového argumentu na~příkazové řádce, tak je nutné jej také
správně naimplementovat do~metody createParser.

\subsection{Změna použitého databázového systému}

V~důsledku dlouhého rozhodování se o~tom, jak se budou informace o~výsledcích měření ukládat vzniklo rozhraní Database.
Rozhraní má spoustu metod. Pokud by se uživatel rozhodl změnit způsob ukládání dat o~měřeních, tak by
musel implementovat celé toto rozhraní. Po~implementaci rozhraní pak už jen stačí změnit metodu constructDatabase ve~třídě
SetupUtilities, která vrací instanci objektu typu Database.

\subsection{Rozšíření o příkaz}

Každý příkaz PerfEval se skládá ze~dvou tříd. Jedná se o~třídu implementující rozhraní CommandSetup
a~o~třídu implementující rozhraní Command. Pokud by uživatel chtěl doplnit nějaký nový příkaz,
který by v~kontextu systému dával smysl, tak je to možné. Dobrý příklad pro~reprezentování nového příkazu
bude vyhodnocení výkonu s~grafickým výstupem.

Na~rozdíl od~stávajícího vyhodnocování by~grafické vyhodnocování potřebovalo údaje z~více než dvou posledních verzí.
Proto by příkaz evaluate-graphical třída rozpoznala jako a~vytvořila instanci třídy EvaluateGraphicalSetup.
Na~této třídě, implementaci CommandSetup, by pak zavolala metodu setup. Metoda setup na~třídě EvaluateGraphicalSetup
by měla za~úkol z~konfigurace PerfEvalu a~příkazové řádky vyrobit objekt EvaluateGraphicalCommand. Tento objekt, který
by implementoval rozhraní Command, by pak metoda getCommand na~třídě Parser vrátila.

Zbytek programu by se již nezměnil metoda main ve~třídě Main by vykonala dodaný příkaz. Spustila by~standardním způsobem
metodu execute na~instanci objektu typu Command.

Zaregistrování nového příkazu by probíhalo přidáním nové položky do~statického konstruktoru třídy Parser.
Položka mapy commandPerSetup má obdobnou strukturu jako v~případě rozšíření o~nový MeasurementParser.
Dodal by se řádek s~názvem příkazu typu String a~z~reference na~bezparametrický konstruktor
implementace rozhraní CommandSetup. Název příkazu odpovídá příkazu, kterým se bude volat z~příkazové řádky.

\subsection{Omezená rošiřitelnost ve vyhodnocování}

Jeden z nejhorších požadavků na rozšíření systému by bylo rozšíření v oblasti vyhodnocování.
Jedná se zejména o změnu implementace třídy PerformanceEvaluator. Tato třída utváří
celkovou vyhodnocovací logiku. Její rozšiřitelnost je omezená na implementace rozhraní
StatisticTest.

Omezenou změnu chování vyhodnocování výkonnostních testů provést lze. Omezené změny ve~vyhodnocování
se provádí změnami hodnot uvnitř config.ini souboru.