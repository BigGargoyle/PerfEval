\chapter*{Závěr}
\addcontentsline{toc}{chapter}{Závěr}

Tato práce se snaží navázat na využívání unit testů při vývoji softwaru.
Nástroj PerfEval, který byl vytvořen, má za úkol poskytnout vývojářům možnost
psát výkonnostní testy obdobně jako unit testy.

V první kapitole bylo vysvětleno, co je průběžná integrace. Dále byl popsán scénář programátora, který
ji využívá pro udržování korektnosti svého kódu, a jak by ji chtěl využít i pro udržování výkonu.
Programátor měl napsané výkonnostní testy pomocí benchmarkovacího frameworku a potřeboval je vyhodnocovat.
Neměl ale žádný nástroj, který by uměl výsledky měření porovnat a vyhodnotit změnu výkonu.

Druhá kapitola se věnuje analýze problematiky kolem vyhodnocování výkonu softwaru. Je zde popsáno jak vypadají výstupy benchmarkovacích frameworků
JMH a BenchmarkDotNET. Dále jsou popsány statistické metody, které se následně v nástroji využívají k porovnání
výsledků měření.

Ve třetí kapitole je vysvětleno několik důležitých rozhodnutí, které byly provedeny před a v průběhu vývoje nástroje PerfEval.
Zároveň byla nastíněna architektura nástroje a jakým způsobem se používá.

Uživatelská dokumentace ukazuje jaké příkazy má uživatel k~dispozici včetně jejich argumentů.
V~této kapitole se také uživatel dozví jak nástroj instalovat a používat včetně toho, jak jej konfigurovat.

Programátorská dokumentace obsahuje detailní popis struktury kódu. Je zde popsáno jaké byly využité návrhové vzory
a jak spolu jednotlivé třídy vzájemně interagují.

Výsledkem práce je nástroj PerfEval jehož způsob použití včetně nasazení v rámci existujícího projektu je demonstrováno
v poslední kapitole. Nástroj byl nasazen v rámci projektu Crate a byl použit k analýze výkonnostních testů verzí, které byly
označené, jako verze se změnou výkonem.

Nástroj změnu výkonu detekoval a označil ji jako významnou. Zároveň byl nástroj schopen detekovat také zlepšení výkonu.
Nástroj PerfEval byl tedy úspěšně nasazen v rámci projektu Crate a splnil tak očekávání, která na něj byla kladena.
Nástroj Perfeval tedy může být používán v rámci průběžné integrace spolu s unit testy a dalšími nástroji pro udržování kvality kódu.