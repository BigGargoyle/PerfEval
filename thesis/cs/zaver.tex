\chapter{Závěr}
%\addcontentsline{toc}{chapter}{Závěr}

Tato práce se snaží navázat na využívání unit testů při vývoji softwaru.
Nástroj PerfEval, který byl vytvořen, má za úkol poskytnout vývojářům možnost
psát výkonnostní testy obdobně jako unit testy.

V první kapitole bylo vysvětleno, co je průběžná integrace. Dále byl popsán scénář programátora, který
ji využívá pro udržování korektnosti svého kódu, a~jak by ji chtěl využít i~pro udržování výkonu.
Programátor měl napsané výkonnostní testy pomocí benchmarkovacího frameworku a~potřeboval je vyhodnocovat.
Neměl ale žádný nástroj, který by uměl výsledky měření porovnat a~vyhodnotit změnu výkonu.

Druhá kapitola se věnuje analýze problematiky kolem vyhodnocování výkonu softwaru. Je zde popsáno jak vypadají výstupy benchmarkovacích frameworků
JMH a BenchmarkDotNET. Dále jsou popsány statistické metody, které se následně v~nástroji využívají k~porovnání
výsledků měření.

Ve třetí kapitole je vysvětleno několik důležitých rozhodnutí, které byly provedeny před a~v~průběhu vývoje nástroje PerfEval.
Zároveň byla nastíněna architektura nástroje a~jakým způsobem se používá.

Uživatelská dokumentace ukazuje jaké příkazy má uživatel k~dispozici včetně jejich argumentů.
V~této kapitole se také uživatel dozví jak nástroj instalovat a~používat včetně toho, jak jej konfigurovat.

Programátorská dokumentace obsahuje detailní popis struktury kódu. Je zde popsáno jaké byly využité návrhové vzory
a~jak spolu jednotlivé třídy vzájemně interagují.

Výsledkem práce je nástroj PerfEval jehož způsob použití včetně nasazení v~rámci existujícího projektu je demonstrováno
v~poslední kapitole. Nástroj byl nasazen v~rámci projektu Crate a~byl použit k~analýze výkonnostních testů verzí, které byly
označené, jako verze se změnou výkonu.

Ukázalo se, že vyvinutý nástroj PerfEval je schopen detekovat změny výkonu. Při použití v~rámci
průběžné integrace je vhodné dodávat verzi, která bude považovaná za~referenční. Tato verze by měla
mít výkon, který chce uživatel udržet. V~případě, že se výkon změní, tak PerfEval uživatele upozorní.
Zároveň tento nástroj splňuje požadavek, že se má připodobnit k~používání unit testů.

Tento nástroj tedy řeší původní programátorův problém chybějícího nástroje pro vyhodnocování výkonu.
Dodáním řádku do svého měřícího skriptu programátor předá informace o~výsledcích měření nástroji PerfEval.
Následně programátor jedním příkazem vyhodnotí změnu mezi požadovanými verzemi.

Do~budoucna by bylo možné nástroj rozšířit například o~vyhodnocování výkonu na bázi skóre.
Optimalizace výkonu totiž obvykle bývá taková, že zlepšením výkonu jednoho testu se zhorší výkon jiného.
Proto by se jednotlivým testům mohlo přiřazovat skóre, které by bylo závislé na změně výkonu a~na tom, jak moc je test důležitý.
Pomocí tohoto rozšíření by pak mohl PerfEval uživateli podat informaci o~tom jestli je zjištěná změna výkonu
pozitivní nebo negativní.