\chapter{Uživatelská dokumentace systému PerfEval}

\section{Rychlý start}

V~této kapitole bude v~několika stručných krocích vysvětleno, jak nainstalovat aplikaci PerfEval a~porovnat pomocí
ní výkonu dvou verzí softwaru. Rychlý start je určen pro uživatele Linuxových distribucí. Na~platformě Windows je
vhodné využít WSL (Windows Subsystem for Linux) nebo jiný nástroj, který umožní spouštění Linuxových příkazů.

\begin{enumerate}
    \item V~adresáři se zdrojovým kódem spusťte příkaz \texttt{./gradlew jar}. Tento příkaz zkompiluje a~sestaví aplikaci PerfEval.
    \item Přejděte do adresáře svého projekt, kde chcete PErfeval používat.
    \item Zadejte příkaz \texttt{./perfeval.sh init -\--benchmark-parser parser-name}. \newline V~případě, že nevíte jaký parser použít, nezadávejte tento argument
        a~systém vám nabídne dostupné parsery. Parser vyberte podle použitého testovacího frameworku a~výstupního formátu.
        Po vybrání parseru zadejte jeho název jako argument parametru \texttt{-\--benchmark-parser}.
    \item Přidejte výsledky měření referenční verze pomocí příkazu \texttt{./perfeval.sh index-new-result -\--path path-to-results -\--version 1.0}.
    \item Přidejte výsledky měření nové verze pomocí příkazu \texttt{./perfeval.sh \newline index-new-result -\--path path-to-results -\--version 2.0}.
    \item Porovnejte výsledky měření pomocí příkazu \texttt{./perfeval.sh evaluate}.
    \item Na standardní výstup bude vypsána tabulka s~výsledky porovnání verzí 1.0 a~2.0. Pokud došlo k~zhoršení výkonu, bude program ukončen s~nenulovým exit kód.
    
\end{enumerate}


\section{Instalace}

Instalace systému PerfEval probíhá pomocí nástroje gradle. V~adresáři se zdrojovým kódem se nachází
spustitelný soubor \texttt{gradlew}, který je určen pro kompilaci a~sestavení PerfEvalu v~systémech Linux.
Pro kompilaci a~sestavení v~systémech Windows je určený spustitelný soubor \texttt{gradlew.bat}.

\begin{itemize}
    \item Instalace v systému Linux: \lstinline|./gradlew jar|
    \item Instalace v systému Windows: \lstinline|gradlew.bat jar|
\end{itemize}

Po kompilaci a~sestavení pomocí zmíněných souborů vznikne soubor typu JAR v~podadresáři \texttt{build/libs}.
Soubor je možné spustit pomocí příkazu \texttt{java -jar PerfEval.jar}, nebo pomocí připravených skriptů
\texttt{perfeval.sh} pro systémy Linux a \texttt{perfeval.bat} pro systémy Windows. Tyto skripty se nachází
taktéž v~adresáři se zdrojovým kódem.

\begin{itemize}
    \item Spuštění v systému Linux: \lstinline|./perfeval.sh|
    \item Spuštění v systému Windows: \lstinline|perfeval.bat|
\end{itemize}

\section{Dostupné příkazy}

Tato část práce se~zabývá jednotlivými příkazy systému PerfEval a jejich parametry.
V~jednotlivých podkapitolách je vysvětleno, k~čemu se daný příkaz používá. V~podkapitolách
se nachází také informace o~volitelných a~povinných parametrech jednotlivých příkazů.

\subsection{Příkaz init}

Příkaz init slouží k~inicializaci systému v~rámci aktuálního pracovního adresáře.
Systém PerfEval po~spuštění hledá v~pracovním adresáři složku s~názvem .performance. Pokud složka
není nalezena a~nebyl zadán příkaz init, končí s~chybou, že systém není inicializovaný.

\subsection*{Povinné argumenty:}
\begin{itemize}[label=\texttt{\textbf{\textendash}}]
    \item[\texttt{benchmark-parser}] Nastavení parseru, který se bude pro tento projekt používat.
        Jméno parseru je zadávané jako parametr tohoto příznaku.
        Parser se volí podle použitého testovacího frameworku a~výstupního formátu.
        Jedná se o~jediný parametr konfiguračního souboru, který lze zadat při inicializaci.
        Důvodem je, že je to jediný konfigurační parametr, jehož hodnotu není možné nastavit nějakým výchozím způsobem.
        Určuje totiž podobu a~formát vstupních dat, které PerfEval očekává.
\end{itemize}

\subsection*{Volitelné argumenty:}
\begin{itemize}[label=\texttt{\textbf{\textendash}}]
    \item[\texttt{force}] Příznak, který vynutí inicializaci i v případě, že je systém v adresáři již inicializovaný.
\end{itemize}

\subsection{Příkaz index-new-result}

Příkaz index-new-result slouží k~přidání výsledků měření výkonu do~databáze. Databáze
je pro~každý projekt zvlášť a~je tedy možné na~jednom zařízení systémem PerfEval spravovat více projektů.
Při přidávání informací o~souboru s~výsledky je nutné zadat cestu k tomuto souboru.
Verze, ke~které byly výsledky změřeny, může být zadaná také, nebo ji systém zkusí určit
podle git repozitáře.

\subsection*{Povinné argumenty:}
\begin{itemize}[label=\texttt{\textbf{\textendash}}]
    \item[\texttt{path}] Parametr tohoto příznaku udává cestu k~souboru s~výsledky měření.
\end{itemize}

\subsection*{Volitelné argumenty:}
\begin{itemize}[label=\texttt{\textbf{\textendash}}]
    \item[\texttt{version}] Parametr tohoto příznaku udává textovou reprezentaci verze SW, která se měřila.
    \item[\texttt{tag}] Parametr tohoto příznaku udává tag verze měření.
\end{itemize}

\subsection{Příkaz index-all-results}

Příkaz index-all-results slouží k~přidání více výsledků měření výkonu do~databáze.
Při přidávání informací o~souborech s~výsledky je nutné zadat cestu k~adresáři s~těmito výsledky.
Budou přidány všechny (i~zanořené) soubory v~tomto adresáři.
Verze, ke~které byly výsledky změřeny, může být zadaná také, nebo ji systém zkusí odhadnout
podle git repozitáře.

\subsection*{Povinné argumenty:}
\begin{itemize}[label=\texttt{\textbf{\textendash}}]
    \item[\texttt{path}] Parametr tohoto příznaku udává cestu k adresáři s výsledky měření.
\end{itemize}

\subsection*{Volitelné argumenty:}
\begin{itemize}[label=\texttt{\textbf{\textendash}}]
    \item[\texttt{version}] Parametr tohoto příznaku udává textovou reprezentaci verze SW, která se měřila.
    \item[\texttt{tag}] Parametr tohoto příznaku udává tag verze měření.
\end{itemize}

\subsection{Příkaz evaluate}

Příkaz evaluate porovnává dvě poslední zaznamenané verze, které byly změřeny.
Verze k~porovnání je možné specifikovat také manuálně pomocí příznaků. Výstupem
je tabulka nebo JSON s~výsledky porovnání. V~případě, že alespoň u~jednoho porovnání
došlo ke~zhoršení výkonu, bude selhání signalizováno exit kódem 1.

\subsection*{Volitelné argumenty:}
\begin{itemize}[label=\texttt{\textbf{\textendash}}]
    \item[\texttt{new-version}] Parametr udává textovou reprezentaci verze, která se má při porovnání považovat za~novější.
    \item[\texttt{new-tag}]     Pouze soubory s~tímto tagem budou použity k~porovnání jako novější.
    \item[\texttt{old-version}] Parametr udává textovou reprezentaci verze, která se má při porovnání považovat za~starší.
    \item[\texttt{old-tag}]     Pouze soubory s~tímto tagem budou použity k~porovnání jako starší.
    \item[\texttt{t-test}]      T-test bude při statistickém porovnání použitý místo bootstrapu.
    \item[\texttt{json-output}] Formát výstupu bude JSON.
    \item[\texttt{html-output}] Výstup bude ve formátu HTML stránky.
    \item[\texttt{html-template}] Při použití příznaku html-output je možné ještě jako argument tohoto příznaku dodat adresu nové HTML šablony pro vypsání výsledků.
    \item[\texttt{junit-xml-output}] Výstup bude ve formátu JUNIT XML. Tento formát je běžně přijímaný nástrojem Git v~rámci průběžné integrace.
    \item[\texttt{filter}]      Parametr tohoto příznaku umožňuje filtrovat výsledky podle názvu testovací metody.
                                Možnosti jsou \texttt{test-id}, \texttt{size-of-change}, \texttt{test-result}.
\end{itemize}

\subsection{Příkaz list-undecided}

Příkaz list-undecided porovnává dvě poslední zaznamenané verze, které byly změřeny.
Verze k~porovnání je možné specifikovat také manuálně pomocí příznaků. Výstupem
jsou dva sloupce oddělené znakem tabulátoru. V~prvním sloupci je název testovací metody.
Ve~druhém sloupci je počet měření, které jsou potřeba, aby bylo možné s~dostatečnou
pravděpodobností říct, že je výkon stejný.

\subsection*{Volitelné argumenty:}
\begin{itemize}[label=\texttt{\textbf{\textendash}}]
    \item[\texttt{new-version}] Parametr udává textovou reprezentaci verze, která se má při porovnání považovat za~novější.
    \item[\texttt{new-tag}]     Pouze soubory s~tímto tagem budou použity k~porovnání jako novější.
    \item[\texttt{old-version}] Parametr udává textovou reprezentaci verze, která se má při porovnání považovat za~starší.
    \item[\texttt{old-tag}]     Pouze soubory s~tímto tagem budou použity k~porovnání jako starší.
    \item[\texttt{t-test}]      T-test bude při statistickém porovnání použitý místo bootstrapu.
    \item[\texttt{filter}]      Parametr tohoto příznaku umožňuje filtrovat výsledky podle názvu testovací metody.
                                Možnosti jsou \texttt{test-id}, \texttt{size-of-change}, \texttt{test-result}.
\end{itemize}

\subsection{Příkaz list-results}

Příkaz list-results vypíše informace o~souborech uložených v databázi. Příkaz nemá
žádné argumenty. Výstupní formát je tabulka s informacemi o souborech. Poskytuje jednoduchý
přehled o~souborech v~databázi PerfEvalu.

\section{Konfigurační soubor}

Po spuštění systému PerfEval s~příkazem \texttt{init} dojde k~vytvoření složky .performance
v~pracovním adresáři. Ve~vytvořené složce bude konfigurační soubor config.ini.
Tento soubor obsahuje nastavení spojené s~používáním systému PerfEval při~vyhodnocování
výkonu jednoho projektu. Změnou hodnot v~konfiguračním souboru je možné omezeně změnit chování
systému PerfEval.

\subsection*{Hodnoty v konfiguračním souboru}
\begin{itemize}[label=\texttt{\textbf{\textendash}}]
    \item[\texttt{falseAlarmProbability}]       Určuje pravděpodobnost chyby I. druhu při testování hypotézy, že výkony verzí jsou stejné.
    \item[\texttt{accuracy}]      Určuje maximální relativní šířku intervalu spolehlivosti
    \item[\texttt{minTestCount}]    Určuje minimální počet testů (běhů), který bude dodán.
    \item[\texttt{maxTestCount}]    Určuje maximální počet testů (běhů), který je uživatel schopný změřit.
    \item[\texttt{tolerance}]       Určuje maximální pokles výkonu (relativně vůči starší verzi), který nezpůsobí selhání.
    \item[\texttt{git}]             Určuje, jestli projekt podléhá správě verzí pomocí nástroje git. Nabývá hodnot TRUE a~FALSE.
    \item[\texttt{parserName}]      Jméno parseru, který bude použit při zpracovávání souborů s~výsledky měření.
    \item[\texttt{highDemandOfRuns}] Určuje, jestli má PerfEval hlásit, že je zapotřebí vyšší počet běhů, než udává maxTestCount. Nabývá hodnot TRUE a~FALSE.
    \item[\texttt{improvedPerformance}] Určuje, jestli má PerfEval hlásit, že výkon novější verze je lepší.
\end{itemize}
