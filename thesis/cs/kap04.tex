\chapter{Uživatelská dokumentace systému PerfEval}

\section{Instalace}

TODO: dodat, až se to bude vědět

\section{Dostupné příkazy}

Tato část práce se~zabývá jednotlivými příkazy systému PerfEval a jejich parametry.
V~jednotlivých podkapitolách je vysvětleno k~čemu se daný příkaz používá. V~podkapitolách
se nachází také informace o~volitelných a~povinných parametrech jednotlivých příkazů.

\subsection{Příkaz init}

Příkaz init slouží k~inicializaci systému v~rámci aktuálního pracovního adresáře.
Systém PerfEval po spuštění hledá v pracovním adresáři složku .performance. Pokud složka
není nalezena a~nebyl zadán příkaz init, tak končí s~chybou, že systém není inicializovaný.

\subsection*{Povinné argumenty:}
\begin{itemize}[label=\texttt{\textbf{\textendash}}]
    \item[\texttt{benchmark-parser}] Nastavení parseru, který se bude pro tento projekt používat.
        Jméno parseru je zadávané jako parametr tohoto příznaku.
        Parser se volí podle použitého testovacího frameworku a~výstupního formátu.
\end{itemize}

\subsection*{Volitelné argumenty:}
\begin{itemize}[label=\texttt{\textbf{\textendash}}]
    \item[\texttt{force}] Příznak, který vynutí inicializaci i v případě, že je systém v adresáři již inicializovaný.
\end{itemize}

\subsection{Příkaz index-new-result}

Příkaz index-new-result slouží k~přidání výsledků měření výkonu do~databáze. Databáze
je pro~každý projekt zvlášť a~je tedy možné na~jednom zařízení systémem PerfEval spravovat více projektů.
Při přidávání informací o souboru s výsledky je nutné zadat cestu k tomuto souboru.
Verze ke~které byly výsledky změřeny může být zadaná také, nebo ji systém zkusí odhadnout
podle git repozitáře.

\subsection*{Povinné argumenty:}
\begin{itemize}[label=\texttt{\textbf{\textendash}}]
    \item[\texttt{path}] Parametr tohoto příznaku udává cestu k~souboru s~výsledky měření.
\end{itemize}

\subsection*{Volitelné argumenty:}
\begin{itemize}[label=\texttt{\textbf{\textendash}}]
    \item[\texttt{version}] Parametr tohoto příznaku udává textovou reprezentaci verze SW, která se měřila.
    \item[\texttt{tag}] Parametr tohoto příznaku udává tag verze měření.
\end{itemize}

\subsection{Příkaz index-all-results}

Příkaz index-all-results slouží k~přidání více výsledků měření výkonu do~databáze.
Při přidávání informací o~souborech s~výsledky je nutné zadat cestu k~adresáři s~těmito výsledky.
Budou přidány všechny (i~zanořené) soubory v~tomto adresáři.
Verze ke~které byly výsledky změřeny může být zadaná také, nebo ji systém zkusí odhadnout
podle git repozitáře.

\subsection*{Povinné argumenty:}
\begin{itemize}[label=\texttt{\textbf{\textendash}}]
    \item[\texttt{path}] Parametr tohoto příznaku udává cestu k adresáři s výsledky měření.
\end{itemize}

\subsection*{Volitelné argumenty:}
\begin{itemize}[label=\texttt{\textbf{\textendash}}]
    \item[\texttt{version}] Parametr tohoto příznaku udává textovou reprezentaci verze SW, která se měřila.
    \item[\texttt{tag}] Parametr tohoto příznaku udává tag verze měření.
\end{itemize}

\subsection{Příkaz evaluate}

Příkaz evaluate porovnává dvě poslední zaznamenané verze, které byly změřeny.
Verze k~porovnání je možné specifikovat také manuálně pomocí příznaků. Výstupem
je tabulka, nebo JSON s~výsledky porovnání. V~případě, že alespoň u~jednoho porovnání
došlo ke~zhoršení výkonu, bude selhání signalizováno exit kódem 1.

\subsection*{Volitelné argumenty:}
\begin{itemize}[label=\texttt{\textbf{\textendash}}]
    \item[\texttt{new-version}] Parametr udává textovou reprezentaci verze, která se má při porovnání považovat jako novější.
    \item[\texttt{new-tag}]     Pouze soubory s~tímto tagem budou použity k~porovnání jako novější.
    \item[\texttt{old-version}] Parametr udává textovou reprezentaci verze, která se má při porovnání považovat jako starší.
    \item[\texttt{old-tag}]     Pouze soubory s~tímto tagem budou použity k~porovnání jako starší.
    \item[\texttt{t-test}]      T-test bude při statistickém porovnání použitý místo bootstrapu.
    \item[\texttt{json-output}] Formát výstupu bude JSON.
    \item[\texttt{html-output}] Výstup bude ve fromátu HTML stránky.
    \item[\texttt{html-template}] Při použití příznaku html-output je možné ještě jako argument tohoto příznaku dodat novou HTML šablonu pro vypsání výsledků.
\end{itemize}

\subsection{Příkaz list-undecided}

Příkaz evaluate porovnává dvě poslední zaznamenané verze, které byly změřeny.
Verze k~porovnání je možné specifikovat také manuálně pomocí příznaků. Výstupem
jsou dva sloupce oddělené znakem tabulátoru. V~prvním sloupci je název testovací metody.
Ve~druhém sloupci je počet měření, které jsou potřeba, aby bylo možné s~dostatečnou
pravděpodobností říct, že je výkon stejný.

\subsection*{Volitelné argumenty:}
\begin{itemize}[label=\texttt{\textbf{\textendash}}]
    \item[\texttt{new-version}] Parametr udává textovou reprezentaci verze, která se má při porovnání považovat jako novější.
    \item[\texttt{new-tag}]     Pouze soubory s~tímto tagem budou použity k~porovnání jako novější.
    \item[\texttt{old-version}] Parametr udává textovou reprezentaci verze, která se má při porovnání považovat jako starší.
    \item[\texttt{old-tag}]     Pouze soubory s~tímto tagem budou použity k~porovnání jako starší.
    \item[\texttt{t-test}]      T-test bude při statistickém porovnání použitý místo bootstrapu.
\end{itemize}

\subsection{Příkaz list-results}

Příkaz list-results vypíše informace o~souborech uložených v databázi. Příkaz nemá
žádné argumenty. Výstupní formát je tabulka s informacemi o souborech. Poskytuje jednoduchý
přehled o~souborech v~databázi PerfEvalu.

\section{Konfigurační soubor}

Po spuštění systému PerfEval s~příkazem init dojde k~vytvoření složky .performance
v~pracovním adresáři. Ve~vytvořené složce bude konfigurační soubor config.ini.
Tento soubor obsahuje nastavení spojené s~používáním systému PerfEval při~vyhodnocování
výkonu jednoho projektu. Změnou hodnot v~konfiguračním souboru je možné omezeně změnit chování
systému PerfEval.

\subsection*{Hodnoty v konfiguračním souboru}
\begin{itemize}[label=\texttt{\textbf{\textendash}}]
    \item[\texttt{critValue}]       Určuje pravděpodobnost chyby I. druhu při testování hypotézy, že výkony verzí jsou stejné.
    \item[\texttt{maxCIWidth}]      Určuje maximální relativní šířku intervalu spolehlivosti
    \item[\texttt{minTestCount}]    Určuje minimální počet testů (běhů), který bude dodán.
    \item[\texttt{maxTestCount}]    Určuje maximální počet testů (běhů), který je uživatel schopný změřit.
    \item[\texttt{tolerance}]       Určuje maximální pokles výkonu (relativně vůči starší verzi), který nezpůsobí selhání.
    \item[\texttt{git}]             Určuje jestli projekt podléhá správě verzí pomocí technologie git. Nabývá hodnot TRUE a~FALSE.
    \item[\texttt{parserName}]      Jméno parseru, který bude použit při zpracovávání souborů s~výsledky měření.
\end{itemize}
