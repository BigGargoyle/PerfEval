%%% Šablona pro jednoduchý soubor formátu PDF/A, jako treba samostatný abstrakt práce.

\documentclass[12pt]{report}

\usepackage[a4paper, hmargin=1in, vmargin=1in]{geometry}
\usepackage[a-2u]{pdfx}
\usepackage[czech]{babel}
\usepackage[utf8]{inputenc}
\usepackage[T1]{fontenc}
\usepackage{lmodern}
\usepackage{textcomp}

\begin{document}

%% Nezapomeňte upravit abstrakt.xmpdata.

Při vývoji softwaru se běžně používají unit testy. Tato práce navazuje na~tento
zvyk a~výsledný nástroj PerfEval podobným způsobem provádí testování výkonu.

Nástroj PerfEval má za~úkol porovnat výsledky měření výkonu dvou verzí softwaru
a~vyhodnotit jestli je výkon novější verze horší. Nástroj využívá výsledky měření
běžných benchmarkovacích frameworků. Porovnáním výsledků měření těchto frameworků
pomocí statistických metod zjišťuje změny výkonu mezi verzemi.

Stejně tak jako unit testy upozorní uživatele, pokud jeho kód není korektní,
tak i~PerfEval upozorňuje uživatele na změnu výkonu.
PerfEval je konzolová aplikace, a~tudíž je možné jej ovládat jednoduše ze~skriptů stejně jako unit testy.


\end{document}
