\chapter{Kontext}

\section{Testování softwaru při vývoji}
Při vývoji softwaru je vhodné vyvíjený software neustále testovat. Software lze testovat
více způsoby, ale cílem testů je vždy otestovat některý z~důležitých kvalitativních
atributů jako je například korektnost, nebo výkon.

Pro testování korektnosti se obvykle používají unit testy. Jedná se většinou o~krátké testovací
funkce, které kontrolují jestli se testovaný kód chová požadovaným způsobem. Zjišťují tedy jestli
kód na zadaný vstup vrátí očekávaný výstup. Ke~kódu který není aktuálně vyvíjen, se obvykle nemění
ani unit testy, a~proto umožňují udržovat neustálou korektnost již hotového kódu. Zda-li je kód
korektní se pomocí unit testů zjistí velmi jednoduše. Kód je korektní pokud vrátil očekávaný
výstup.

Testování výkonu obvykle probíhá tak, že se použije nějaký vhodný měřící framework.
Obvyklé frameworky pro měření výkonu mají vlastní pravidla, jak se mají oanotovat metody, které
se mají měřit. Tyto frameworky umožňují měřit výkon softwaru podle různých metrik. Mezi tyto metriky
se řadí čas, propustnost a~například spotřeba paměti. Výsledky měření frameworky umí obvykle zaznamenat jak do~strojově
čitelného formátu, tak do formátu čitelného pro člověka. Výsledek měření je ale pouze sada čísel, kde
jsou ke~jménům testovaných metod přiřazeny naměřené hodnoty.

TODO: doplnit obrázek nějakého výstupu například BenchmarkDotNet, nebo JMH?
možná by se hodilo nějak vhodně zasazený přímo textový výstup

Výsledky testování výkonu se tedy musejí vyhodnocovat tak, že se podrobně prozkoumá výsledná sada čísel.
Oproti testování korektnosti, kdy test projde, nebo neprojde je testování výkonu výrazně složitější.
Pohledem na samostatnou sadu dat se nedá určit zda-li je software dostatečně rychlý. Aby bylo možné
ze~sady určit něco vypovídajícího mohl by být stanoven limit výkonnosti. Například by se stanovilo, že
metoda se nebude konat déle než 5\,s. Tento přístup nemusí být vypovídající při dlouhodobém vývoji
a~při časech hluboko pod limitem. Proto je vhodné aby se datové sady, které testovací framework produkuje,
testovaly navzájem mezi sebou. Porovnáním sad je totiž možné zjistit, jestli nedošlo k~významným změnám výkonu
při~vývoji od~předchozí verze. Frameworky samotné však tuto možnost obvykle nemají.

\section{Průběžná integrace}
Při vývoji softwaru se čato používají nástroje pro automatizování některých činností.
Obvykle se automatizují činnosti jako jsou správa verzí, spouštění testů a jejich vyhodnocování.

Pro správu verzí při vývoji softwaru se obvykle používá technologie git.
Jedná se o sytém pro správu verzí. Git je vhodný i pro práci ve velkých týmech.
Umožňuje totiž členění projektu do větví. Každá větev je vhodná pro vývoj samostné části aplikace
a následným spojováním větví pak dochází k propojení funkčností vyvinutých v jednotlivých větvích.
Technologie si formou pamatování si změn udržuje přehled o průběžných verzích a umožňuje uživatelům (programátorům)
vrátit se vývoji zpět. Technologii je možné ovládat jednoduchými příkazy z příkazové řádky.
Je tedy vhodná pro psaní krátkých skriptů.

Nástroj git umí jednoduše zpostředkovat průběžnou integraci (continuous integration). Průběžná integrace umožňuje dělat uživateli automatizované
kroky při nahrávání nových verzí do~větve nebo při spojování větví. Při průběžné integraci tedy jde o~spouštění
skriptu při některé ze zmíněných událostí. V rámci průběžné integrace je možné testovat software pomocí unit testů
i~benchmarků pro měření výkonu. Dále je možné použít jakékoli jiné příkazy příkazové řádky.
Do průběžné integrace je tedy možné jednoduše zapojit téměř jakoukoli konzolovou aplikaci.

\section{Scénář}
%%% následující odstavce pojednávají o konkrétním problému, který se při testování výkonu řeší
Mějme programátora, který se rozhodne vyvíjet nový software. Projekt se mu začne rozrůstat, a~tak
aby mohl udržovat větší projekt napíše si unit testy. Pomocí unit testů si udržuje průběžnou korektnost
již hotového kódu. Dále pojme podezření, že se jeho stávající kód začíná zpomalovat (klesá výkon). Proto se
rozhodne použít framework pro testování výkonu a spustí testy pro měření výkonu svého kódu.

Programátor změří výkon na verzi před tím než pojmul podezření o zpomalování. Následně změří výkon aktuální verze.
Výsledkem těchto měření jsou dvě tabulky s daty. V každé z nich je jméno testovací metody k níž je přiřazen výsledek
měření. Protože ale benchmarkovací framework funguje jako stopky, tak není schopný porovnat dvě tabulky mezi sebou.
Analogie ke stopkám je, že jsou schopny změřit čas prvního a~druhého kola zvlášť, ale nejsou schopny je porovnat.

Programátor se tedy musí sám podívat, jestli výkon jeho softwaru klesá. Programátor by tedy potřeboval
nástroj, který by uměl vzít výstupy měření z různých verzí a porovnal je mezi sebou. Zároveň by programátor jistě
chtěl, aby byl na~zhoršení výkonu upozorněn podobným způsobem jako na selhání unit testů tj. selháním
průběžné integrace.
