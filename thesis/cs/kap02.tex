\chapter{Problém vyhodnocování výkonu}

\section{Automatické vyhodnocování}
V~minulé kapitole byl zmíněn problém programátora s~testováním výkonu. Programátor má unit
testy, které testují korektnost jeho softwaru. Umí je efektivně spouštět s~každou změnou
pomocí průběžné integrace. Chtěl by, aby mohl podobně efektivně testovat i~výkon svého softwaru.

S~testováním výkonu je ale problém. Když se spustí měření výkonu, tak výsledkem je pouhá
datová sada. Tato datová sada nevypovídá nic o~změně průběžného výkonu, která
je zajímavá. Právě podle změny ve~výkonu softwaru je možné zjistit, jestli je nutné
kód optimalizovat, protože dochází k~významným zhoršením.

Při samotném měření výkonu je ale možné, že se některé hodnoty výrazně odchýlí od ostatních.
Může tedy být nutné měření opakovat a~z~více naměřených hodnot dohromady usuzovat, jak se
výkon v~průběhu vývoje mění. K~analýze takto získaných hodnot se ale hodí využít statistických
metod, které si i~se zmíněnými odchylkami umí poradit.

Vyvinutý systém PerfEval řeší programátorův problém. Jedná se o~nástroj, který může
zakomponovat do~své průběžné integrace tak, aby byl pokles výkonu hlášen. Nástroj
při~spuštění průběžné integrace porovná dvě poslední verze a~případně oznámí zhoršení výkonu.
Na~základě tohoto hlášení může selhat celá průběžná integrace, což je obdobné chování,
jako se očekává při selhání unit testů.

\section{Výsledky měření výkonu}

Z~počátku je dobré se zamyslet nad tím, jak můžou výsledky měření výkonu softwaru vypadat.
První domněnky o~tom, že stačí měřit pouze čas se ukázaly jako mylné. Testovací frameworky
totiž umožňují mimo měření času také měření frekvence nebo například spotřeby paměti.
Předpokládat se tedy dá jen to, že pokud vezmu dva výsledky měření výkonu ze~dvou různých
verzí, tak budou reprezentovány stejnou fyzikální veličinou a~v~lepším případě budou mít
i~stejnou fyzikální jednotku.

Protože testovací frameworky nepoužívají žádné identifikátory testů, tak je nejpřímější
řešení jejich rozpoznávání používat jména testovacích metod jako identifikátor. Tato jména poskytují ve~výsledcích
měření jak framework BenchmarkDotNet, tak framework JMH. Z~dokumentace frameworku Criterion
pro měření výkonu v~jazyce~R se název metody ve~výsledcích nachází také. Z~toho lze
usoudit, že použití jména metody jako identifikátoru může být dostatečně obecné.
(DOKUMENTACE - CRITERION)

\subsection{Výstup měření BenchmarkDotNet}

BenchmarkDotNet je framework určený k~měření výkonu programů na~platformě .NET. V~důsledku
toho, že měří programy na platformě .NET je schopen měřit výkon programů napsaných
v~programovacích jazycích C\#, F\# a~Visual Basic (DOKUMENTACE - BenchmarkDotNet README.md).

Při~měření výkonu pomocí BenchmarkDotNet se měřené hodnoty vypisují na~standardní výstup
včetně konečného shrnutí. Mimo standardní výstup se ještě výsledky měření ukládají
do~strojově zpracovatelných formátu jako je například JSON, nebo CSV.

(DODAT PŘÍLOHU)

Měření výkonu programu v~jazyce C\# pomocí frameworku BenchmarkDotNet má za~výstup již statisticky zpracované hodnoty.
Aby bylo možné sledovat jednotlivé naměřené hodnoty je nutné zvolit již při~psaní testů správný
exportér, který tuto funkci podporuje. Pro~možnost sledování konkrétních naměřených hodnot u~exportéru
nazvaného JsonExporter je nutné nastavit možnost excludeMeasurements na~false, aby se ve~výstupním
souboru hodnoty objevily. Jedna naměřená hodnota se nachází u~výsledku měření jako položka Nanoseconds.
Z~názvu položky lze usoudit, že v~základním nastavení BenchmarkDotNet se měří čas.

Celkově tedy v~souboru s~výsledky nalezneme podrobnosti o~prostředí jako je operační systém,
verze platformy .NET, jméno, typ a~parametry procesoru. Dále se v~souboru s~výsledky nachází
výsledky jednotlivých provedených měření. Výsledek měření u~sebe má informaci o~jméně
testovací metody, zpracované statistické údaje a~naměřené hodnoty z~různých módů a~iterací měření.

Je nutné vybírat jen některé hodnoty, protože C\# je kompilovaný jazyk metodou
JIT. Je tedy nutné dávat pozor jaká data pro~porovnávání výsledků budou použita. Určitě není správně dávat
k~sobě pro porovnání hodnoty, které byly naměřeny před optimalizacemi překladačem a~po~nich. Může totiž dojít
ke~zkreslení výsledků.

TODO: přidat jako přílohu výstupy BenchmarkDotNet odkazovat se do nich a do dokumentace BDN

\subsection{Výstup měření JMH}

JMH je framework pro měření výkonu, který umožňuje pomocí anotací definovat výkonostní testy
pro programy v~jazyce Java. Z~průzkumu (ZDROJ-prezentace) vyplývá, že se jedná o~nejpoužívanější framework
pro měření výkonu pro projekty vyvíjené v jazyce Java. %% https://d3s.mff.cuni.cz/f/resources/icpe2017/slides-unit-testing-performance-in-java-projects.pdf

JMH obdobně jako BenchmarkDotNet poskytuje výsledek měření jako tabulku na standardní výstup.
Dále poskytuje výstup také v~podobě strojově zpracovatelných formátů jako jsou například XML,
nebo JSON. O~výstup v~této podobě je nutné zažádat pomocí argumentů na~příkazové řádce při
spouštění měření (DOKUMENTACE - JMH).

(DODAT PŘÍLOHU)

Ve výstupním souboru měření pomocí JMH lze nalézt informace o~stroji na~kterém probíhalo měření.
Jedná se především o~název stroje a~verzi operačního systému. Dále zde lze nalézt verzi Javy,
ve~které probíhalo měření. V~souboru je možné vidět také ostatní parametry měření, jako je
zahřívací doba a~počet zahřívacích iterací. Zahřívací iterace jsou důležité opět kvůli tomu,
že Java je překládaná technologií JIT.

Jednotlivé naměřené hodnoty jsou ve výstupním souboru dostupné i~ve~výchozím nastavení JMH.
Naměřené hodnoty jsou reprezentovány pomocí metriky. V~rámci metriky jsou uloženy informace
o~fyzikální jednotce, která se měřila, a~sady naměřených hodnot. Pro~každý běh (fork), přičemž běhů
může být v~jednom spuštění JMH více, se~objeví jedna sada měřených hodnot u~metriky.

\section{Použití statistických metod pro analýzu dat}

Pro vyhodnování výkonu je využito metod testování hypotéz. Ve statistickém testování
hypotéz se snažíme zamítnout nulovou hypotézu. V~případě zamítnutí nulové hypotézy
se předpokládá, že platí alternativní hypotéza.

Při testování hypotéz rozeznáváme
chyby I. a~II. druhu. Chyba I. druhu znamená, že jsme nulovou hypotézu zamítli,
i~když platí. Chyba II. druhu znamená, že jsme ji nezamítli, ale ona neplatí.

V našem případě porovnávání výkonu bude nulová hypotéza tvrzení, že výkon obou verzí je stejný.
Jako alternativní hypotézu budeme uvažovat, že výkony obou verzí jsou různé.
Naměřené hodnoty jedné testovací metody každé z verzí budeme považovat za diskrétní náhodnou veličinu.
Pokud hodnoty těchto veličin mají stejné rozdělení, tak budeme tvrdit, že i výkon obou verzí je stejný.

Chyba I. druhu tedy v našem případě znamená, že jsme prohlásili, že výkony dvou verzí jsou různé,
ačkoli jsou stejné. Pravděpodobnost chyby I. druhu je obvyklý parametr statistického testu.
Pravděpodobnost chyby I. druhu bude dále značen jako parametr $\alpha$. Parametr $\alpha$
je součástí konfigurace systému PerfEval.

\subsection{Dvouvýběrový T-test}

Dvouvýběrový T-test se používá při neznalosti rozptylu zkoumaných náhodných veličin.
Protože není možné usuzovat rozptyl naměřených hodnot, tak nám tato vlastnost vyhovuje.
Dvouvýběrový T-test zkoumá jestli je rozdíl středních hodnot $\delta$.
Pokud mají být rozdělení náhodných veličin stejná, rozdíl jejich středních hodnot by měl být nula ($\delta = 0$).

T-test vyžaduje od~náhodných veličin tři vlastnosti. Měly by být nezávislé, normální
a~měly by mít stejný rozptyl. Nezávislost veličin není možné při měření zajistit,
proto bude případná závislost zanedbána. Stejný rozptyl obou veličičn také není možné zaručit.
Lze ale předpokládat, že měření veličin při stejných podmínkách, způsobí podobný rozptyl.
Podle shrnutí o~dvouvýběrových testech (ZDROJ - dvouvýběrové testy) %%-> https://www.karlin.mff.cuni.cz/~turcic/Dvouvyberove_testy.pdf)
způsobí zanedbání shodnosti rozptylů pouze malou chybu.

Posledním z~požadavků dvouvýběrového T-testu je normalita náhodných veličin. Toho docílíme pomocí centrální limitní věty (CLV).
Výsledky měření se skládají z~běhů. Každý běh obsahuje naměřené vzorky náhodné veličiny, která reprezentuje naměřenou hodnotu výkonu.
Pokud budeme v~rámci T-testu namísto naměřených hodnot uvažovat průměry jednotlivých běhů, tak se podle CLV
bude rozdělení těchto průměrů blížit normálnímu rozdělení.

Dvouvýběrový T-test je dnes běžnou součástí standardních matematických knihoven.
Použití T-testu v programu je tedy velmi jednoduché, protože není potřeba zamýšlet se
nad korektností psaného kódu. V~případě systému PerfEval se T-test provádí tak, že
se z~naměřených hodnot vyrobí Studentovo T-rozdělení. Z~tohoto rozdělení se zjistí
intervalový odhad střední hodnoty.

Nakonec se již jen zkoumá zda-li tento interval obsahuje nulu ($\delta = 0$).
Pokud interval nulu neobsahuje, pak lze tvrdit, že s~pravděpodobností $1-\alpha$
nulová hypotéza neplatí.

TODO: odkaz do skript z PaSt1, doplnit nějakou myšlenku? %%-> https://iuuk.mff.cuni.cz/~samal/vyuka/2223/PSt1/skripta.pdf

\subsection{Hierarchický bootstrap}

%% - https://www.youtube.com/watch?v=Xz0x-8-cgaQ

Bootstrap je statistická metoda využívající tzv. resamplování.
Stejně jako u dvouvýběrového T-testu se předpokládá nezávislost náhodných veličin.
Podmínka nezávislosti bude v systému zanedbána, protože ji není možné zaručit.
Bootstrap se jako metoda používá v případě, kdy se o náhodných věličinách nedá
dopředu téměř nic usuzovat. Toto odpovídá případu měření výkonu, kdy neznáme ani
rozptyl ani střední hodnotu.

Mějme naměřený statistický soubor dat o~velikosti n.
Z~tohoto statistického souboru vybereme n-krát náhodný prvek. Z~těchto náhodně vybraných prvků spočítáme průměr.
Spočtený průměr budeme považovat za~nový vzorek nově počítaného statistického souboru.
Namísto malého statistického souboru se tedy vytvoří větší statistický soubor.
Z~nového statistického souboru se určí interval spolehlivosti tak, že se vezme $\frac{\alpha}{2}$-tý
a~$\frac{1-\alpha}{2}$-tý percentil.

Zkoumanou náhodnou veličinou je rozdíl dvou náhodných veličin. Tyto dvě náhodné veličiny jsou dány měřením
výkonnosti dvou verzí softwaru. Nulová hypotéza, která je vyvracena, tvrdí, že obě veličiny mají stejné rozdělení.
Pokud tedy interval spolehlivosti neobsahuje nulu, tak můžeme nulovou hypotézu vyvrátit, protože s~pravděpodobností
$1-\alpha$ neplatí.

Naměřené vzorky však nejsou prostý statistický soubor. Jedná se o~hierarchický soubor dat.
Každé jedno měření se skládá z~jednoho, nebo více běhů. Každý běh se skládá z~jednoho, nebo více naměřených údajů.
Vytváření bootstrapového statistického souboru tedy vypadá trochu odlišně. Prvky tohoto souboru
budou spočteny z~náhodně vybraných běhů. Hodnota prvku náhodně vybraného běhu bude průměr
bootstrapového souboru spočteného z~měření v~tomto běhu. Bootstrapový soubor z~hodnot běhu
se již spočte výše zmíněným způsobem. Vybere se n~náhodných hodnot z~naměřených hodnot v~běhu.

\subsection{Co dělat v případě nevyvrácení hypotézy?}

V~případě nevyvrácení nulové hypotézy nám statistické testy nedávají žádnou informaci.
Nicméně je stále nutné se rozhodnout zda-li nulová hypotéza platí. Je nutné se ale stále
vyvarovat chybě II. druhu. Proto v tomto případě budeme považovat nulovou hypotézu za platnou,
pokud bude interval spolehlivosti dostatečně úzký. Pokud má tedy interval spolehlivosti
dolní mez 1 a horní mez 3, pak je jeho délka 2. Odhadovaný průměr by byl také 2.
Relativní úzkost intervalu by tedy byl poměr průměru a délky, tedy 1.

V případě většího množství vzorků je možné zužovat interval spolehlivosti. Vztah mezi
úzkostí a~počtem vzorků odpovídá $O(\frac{1}{\sqrt{n}})$, kde n je počet vzorků. Z~daného
množství vzorků je tedy možné analyticky určit, kolik vzorků je ještě zapotřebí změřit.

%% https://iuuk.mff.cuni.cz/~samal/vyuka/2223/PSt1/skripta.pdf

V případě, že je interval dostatečně úzký prohlásíme, že nulová hypotéza platí.
V případě, že interval není dostatečně úzký, prohlásíme, že vzorků není dost.
